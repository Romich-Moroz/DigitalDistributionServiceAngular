\section{Анализ литературных источников}
\label{sec:literature_analysis}

\subsection{ASP.NET Core}

ASP.NET Core -- свободно-распространяемый кросс-платформенный фреймворк для создания веб-приложений с открытым исходным кодом. Данная платформа разрабатывается компанией Майкрософт совместно с сообществом и имеет большую производительность по сравнению с ASP.NET. Имеет модульную структуру и совместима с такими операционными системами как Windows, Linux и macOS.

Несмотря на то, что это новый фреймворк, построенный на новом веб-стеке, он обладает высокой степенью совместимости концепций с ASP.NET. Приложения ASP.NET Core поддерживают параллельное управление версиями, при котором разные приложения, работающие на одном компьютере, могут ориентироваться на разные версии ASP.NET Core. Это было невозможно в предыдущих версиях ASP.NET.

В данном курсовом проекте, в соответствии с поставленной задачей, для разработки серверной части веб-приложения будет использоваться ASP.Net Core WebApi, что позволит создать удобный интерфейс для взаимодействия с клиентской частью приложения используя платформу .NET.

\subsection{Angular}
Angular (версия 2 и выше) -- открытая и свободная платформа для разработки веб-приложений, написанная на языке TypeScript, разрабатываемая командой из компании Google, а также сообществом разработчиков из различных компаний. Angular -- полностью переписанный фреймворк от той же команды, которая написала AngularJS.

Предназначен для разработки одностраничных приложений. Его цель -- расширение браузерных приложений на основе MVC-шаблона, а также упрощение тестирования и разработки.

Angular спроектирован с убеждением, что декларативное программирование лучше всего подходит для построения пользовательских интерфейсов и описания программных компонентов, в то время как императивное программирование отлично подходит для описания бизнес-логики. Фреймворк адаптирует и расширяет традиционный HTML, чтобы обеспечить двустороннюю привязку данных для динамического контента, что позволяет автоматически синхронизировать модель и представление. В результате Angular уменьшает роль DOM-манипуляций и улучшает тестируемость.

Angular придерживается MVC-шаблона проектирования и поощряет слабую связь между представлением, данными и логикой компонентов. Используя внедрение зависимости, Angular переносит на клиентскую сторону такие классические серверные службы, как видозависимые контроллеры. Следовательно, уменьшается нагрузка на сервер и веб-приложение становится легче.

В данном курсовом проекте будет использоваться Angular, так как он легко интегрируется с ASP.NET Core и удобен в использовании. Также, Angular является фреймворком для языка TypeScript, обладающего строгой типизацией, что может упростить процесс тестирования приложения.

\subsection{Принципы создания ASP.NET Core WebApi + Angular приложения}
Серверная часть веб-приложения, написанная с использованием ASP.NET Core WebApi имеет следующую структуру:
\begin{itemize}
	\item Класс Startup, в котором настраиваются службы, необходимые приложению и определяется конвейер обработки запросов.
	\item В методе ConfigureServices регистрируются службы для контейнера внедрения зависимостей.
	\item В методе Configure в конвейер обработки запросов встраиваются компоненты промежуточного слоя, в том числе компонент, отвечающий за маршрутизацию.
	\item Определяются маршруты, ведущие к конечным точкам приложения.
	\item Логика обработки запросов описывается в конечных точках -- контроллерам WebApi.
\end{itemize}

Деление веб-приложения на два отдельных проекта -- WebApi для серверной части и Angular для клиентской, может показаться неудобным, ведь это увеличивает объём кода и требует использования разных языков программирования в каждом проекте. Можно воспользоваться и старым подходом, когда всё приложение состоит лишь из сервера, который генерирует и отправляет HTML-страницы в ответ на запросы клиента, однако с развитием веб-приложений стало ясно, что такой подход обладает определённым количеством недостатков, среди которых, например, неизбежная перезагрузка страницы при любом действии пользователя. Всё более предпочтительным вариантом становится создание Single-Page приложений, в которых всё взаимодействие с пользователем выносится в отдельное клиентское приложение, которое посылает запросы на сервер и отображает изменения без перезагрузки веб-страницы, что положительно сказывается на пользовательском опыте.

Основными принципами разработки клиентской части веб-приложения с использованием Angular являются:
\begin{itemize}
	\item Angular-приложения состоят из независимых элементов. Эти элементы называются компонентами, и у каждого компонента своё поведение.
	\item Все компоненты подключаются к главному или дополнительным модулям. Модули управляют компонентами.
	\item Главный модуль контролирует всё приложение, а дополнительные модули следят за работой отдельных элементов.
	\item Для сложных операций вместо компонентов используют сервисы.
	\item Сервисы отвечают только за тот набор логических операций, для которых он предназначен.
\end{itemize}

