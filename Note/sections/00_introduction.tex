\sectioncentered*{Введение}
\addcontentsline{toc}{section}{Введение}
\label{sec:introduction}

В современном мире всё большую популярность набирают онлайн-сервисы, предназначенные для продажи различных товаров или услуг.
Причинами такого быстрого роста популярности являются, в первую очередь, отсутствие необходимости в посещении магазина, а также
экономия финансов на аренду и содержание помещения. Действительно, такая система распространения товаров является выгодной и для клиентов,
и для создателей платформ, а повсеместное распространение и доступность мобильного интернета ещё больше увеличивают 
актуальность онлайн-сервисов.

В качестве темы курсового проекта был выбран "Онлайн-сервис Магазин игр", задачей которого является продажа цифровых копий видеоигр и
организация площадки, где пользователи могут найти информацию и отзывы других пользоватей о интересующих их продуктах. До распространения
онлайн-сервисов, когда человек следил за некоторой интересующей его видеоигрой, чтобы найти какую-нибудь актуальную информацию о процессе
её разработки, ему было необходимо приобрести свежий выпуск игрового журнала и надеяться, что там он найдёт что-то интересное. Для 
приобретения видеоигры также надо было идти в специализированные магазины и покупать копию на физическом носителе.

Очевидно, что такая система обладала множеством недостатков и приводила к большим тратам времени. Также стоит отметить, что физические
носители в последнее время очень сильно утратили свою актуальность, потому что высокоскоростной интернет стал доступным широкому кругу
пользователей. Конечно, нельзя не отметить, что несмотря на всё неудобство и неэффективность такого способа приобритения видеоигр, ещё
остаются люди, для которых он является приоритетным. Это связано с тем, что для них обладание физической копией служит гарантом владения
игрой, и такую логику можно понять. К тому же, некоторые люди занимаются коллекционированием видеоигр, а цифровые копии не способны
удовлетворить такую потребность.

Таким образом, можно отметить, что распространение видеоигр на физических носителях ещё не утратило свою популярность до конца, однако
всё больше и больше людей предпочитают не тратить время на походы в магазины и пользуются сервисами цифровой дистрибуции. Количество
зарегистрированных пользователей в самых популярных сервисах исчисляется сотнями миллионов, поэтому тема курсового проекта обладает
высокой актуальностью и востребованностью.